\documentclass[12pt]{jsarticle}
\usepackage[dvipdfmx]{graphicx}
\textheight = 25truecm
\textwidth = 18truecm
\topmargin = -1.5truecm
\oddsidemargin = -1truecm
\evensidemargin = -1truecm
\marginparwidth = -1truecm

\def\theenumii{\Alph{enumii}}
\def\theenumiii{\alph{enumiii}}
\def\labelenumi{(\theenumi)}
\def\labelenumiii{(\theenumiii)}
\def\theenumiv{\roman{enumiv}}
\def\labelenumiv{(\theenumiv)}
\usepackage{comment}

%%%%%%%%%%%%%%%%%%%%%%%%%%%%%%%%%%%%%%%%%%%%%%%%%%%%%%%%%%%%%%%%
%% sty/ にある研究室独自のスタイルファイル
\usepackage{jtygm}  % フォントに関する余計な警告を消す
\usepackage{nutils} % insertfigure, figref, tabref マクロ

\def\figdir{./figs} % 図のディレクトリ
\def\figext{pdf}    % 図のファイルの拡張子

\begin{document}
%%%%%%%%%%%%%%%%%%%%%%%%%%%%
%% 表題
%%%%%%%%%%%%%%%%%%%%%%%%%%%%
\begin{center}
{\LARGE 平成30年度B4新人研修課題 報告書}
\end{center}

\begin{flushright}
  2018/4/25\\
  高橋 桃花
\end{flushright}
%%%%%%%%%%%%%%%%%%%%%%%%%%%%
%% 概要
%%%%%%%%%%%%%%%%%%%%%%%%%%%%
\section{概要}
\label{sec:introduction}
本資料は平成30年度B4新人研修課題の報告書である.新人研修課題として,SlackBotプログラムを作成した.本資料では,課題内容,課題を通して理解できなかった部分,課題の中で作成できなかった機能,および課題として自主的に作成した機能について述べる.

\section{課題内容}
課題内容は,RubyによるSlackBotプログラムの作成である.なお,本課題で使用するRubyのバージョンは2.5.1である.
\begin{enumerate}
\item 任意の文字列を発言するプログラムの作成
\item SlackBotプログラムへの機能追加
\end{enumerate}

\section{理解できなかった部分}
理解できなかった部分は以下の2点である.

\begin{enumerate}
\item Google Places APIが取得する写真やレビューの順序
\end{enumerate}

\section{作成できなかった機能}
作成できなかった機能は以下のとおりである.

\begin{enumerate}
\item 指定したOutgoing WebHooks以外からのPOSTを拒否する機能
\item 大学や公園など,飲食店以外の場所の情報を,適切なフォーマットで返信する機能
  Google Places APIが返却する情報の項目の中には,場所の属性によっては値が存在しないことがある.たとえば,岡山大学の情報には,価格帯の情報は存在しない.
\item ユーザが入力した場所に関して,現在地を起点に場所を検索する機能
\end{enumerate}

\section{自主的に作成した機能}
以下の機能を自主的に作成した.
\begin{enumerate}
  \item 入力された飲食店の情報を返信する機能
\end{enumerate}

\end{document}
