\documentclass[12pt]{jsarticle}
\usepackage[dvipdfmx]{graphicx}
\textheight = 25truecm
\textwidth = 18truecm
\topmargin = -1.5truecm
\oddsidemargin = -1truecm
\evensidemargin = -1truecm
\marginparwidth = -1truecm

\def\theenumii{\Alph{enumii}}
\def\theenumiii{\alph{enumiii}}
\def\labelenumi{(\theenumi)}
\def\labelenumiii{(\theenumiii)}
\def\theenumiv{\roman{enumiv}}
\def\labelenumiv{(\theenumiv)}
\usepackage{comment}

%%%%%%%%%%%%%%%%%%%%%%%%%%%%%%%%%%%%%%%%%%%%%%%%%%%%%%%%%%%%%%%%
%% sty/ にある研究室独自のスタイルファイル
\usepackage{jtygm}  % フォントに関する余計な警告を消す
\usepackage{nutils} % insertfigure, figref, tabref マクロ

\def\figdir{./figs} % 図のディレクトリ
\def\figext{pdf}    % 図のファイルの拡張子

\begin{document}
%%%%%%%%%%%%%%%%%%%%%%%%%%%%
%% 表題
%%%%%%%%%%%%%%%%%%%%%%%%%%%%
\begin{center}
{\LARGE 平成30年度B4新人研修課題 報告書}
\end{center}

\begin{flushright}
  2018/4/25\\
  高橋 桃花
\end{flushright}
%%%%%%%%%%%%%%%%%%%%%%%%%%%%
%% 概要
%%%%%%%%%%%%%%%%%%%%%%%%%%%%
\section{概要}
\label{sec:introduction}
本資料は,平成30年度B4新人研修課題の報告書である.新人研修課題として,SlackBotプログラムを作成した.本資料では,課題内容,課題を通して理解できなかった部分,課題の中で作成できなかった機能,および課題として自主的に作成した機能について述べる.

\section{課題内容}
RubyによるSlackBotプログラムを作成する.Slack\cite{slack}とは,チャットツールである.また,SlackBotプログラムとは,Slackのチャットにおいて発言したり,ユーザが特定の文字列を入力すると自動で返信したりするプログラムである.なお,本課題で使用するRubyのバージョンは2.5.1である.課題の詳細は以下のとおりである.
\begin{enumerate}
\item 入力された文字列を発言するSlackBotプログラムの作成\\
  SlackのIncoming Webhooksと,Outgoing Webhooksを用いてSlackBotプログラムを作成する.このSlackBotプログラムは,Incoming Webhooksにより発言する機能と,Outgoing Webhooksにより発言を取得した場合,反応する機能をもつ.
\item SlackBotプログラムへの機能追加 \\
  Slack以外のWebサービスのAPIやWebhookを利用した機能を追加する.たとえば,Slack上の発言を契機にして,Slack以外のサービスから天気やニュース情報を取得し,取得した情報を発言する.
\end{enumerate}

\section{理解できなかった部分}
理解できなかった部分は以下の2点である.

\begin{enumerate}
\item Google Places API\cite{placesapi}により返却されるJSONデータの項目のうち,写真,レビューの項目の値の並び順\\
  Google Places APIによりSlackBotプログラムに返却されるJSONデータには,写真,レビューの情報が配列として格納されている.しかし,配列中の情報がどういった順序で並んでいるのかが理解できなかった.
\item Rackの仕組み
\end{enumerate}

\section{作成できなかった機能}
作成できなかった機能は以下のとおりである.

\begin{enumerate}
\item 指定したOutgoing WebHooks以外からのPOSTを拒否する機能
\item JSONデータに値が存在しない項目は,返信内容に表示しない機能 \\
  Google Places APIによりSlackBotプログラムに返却されるJSONデータの項目には,値が存在しない場合がある.今回作成したSlackBotプログラムは,値が存在しない項目は項目名のみを返信する.このため,返信に用いる各項目のうち,JSONデータに値が存在しない項目は返信しないことが考えられる.たとえば,岡山大学と入力する場合,返信項目の一つである価格帯の値は存在しない.この時,返信内容に価格帯の項目は表示しない.
  
\item 現在地を起点に,ユーザが入力した場所を検索する機能\\
  今回作成したSlackBotプログラムは,現在地の位置情報を取得していない.このため,場所を入力する際に,都市名や駅名など,おおよその位置を特定できる情報を併せて入力する必要がある.現在地を起点に場所を検索することで,位置情報を入力することなく場所が特定できる.
\end{enumerate}

\section{自主的に作成した機能}
以下の機能を自主的に作成した.
\begin{enumerate}
\item 入力された飲食店の情報を返信する機能\\
  ユーザが飲食店の名前を入力すると,SlackBotプログラムが飲食店の情報を返信する.返信する情報は,飲食店の名前,開店ステータス,価格帯,評価,公式WebサイトのURL,レビュー,および写真である.
\end{enumerate}

\bibliographystyle{ipsjunsrt}
\bibliography{mybibfile}

\end{document}
