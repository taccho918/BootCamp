\documentclass[12pt]{jsarticle}
\usepackage[dvipdfmx]{graphicx}
\textheight = 25truecm
\textwidth = 18truecm
\topmargin = -1.5truecm
\oddsidemargin = -1truecm
\evensidemargin = -1truecm
\marginparwidth = -1truecm

\def\theenumii{\Alph{enumii}}
\def\theenumiii{\alph{enumiii}}
\def\labelenumi{(\theenumi)}
\def\labelenumiii{(\theenumiii)}
\def\theenumiv{\roman{enumiv}}
\def\labelenumiv{(\theenumiv)}
\usepackage{comment}

%%%%%%%%%%%%%%%%%%%%%%%%%%%%%%%%%%%%%%%%%%%%%%%%%%%%%%%%%%%%%%%%
%% sty/ にある研究室独自のスタイルファイル
\usepackage{jtygm}  % フォントに関する余計な警告を消す
\usepackage{nutils} % insertfigure, figref, tabref マクロ

\def\figdir{./figs} % 図のディレクトリ
\def\figext{pdf}    % 図のファイルの拡張子

\begin{document}
%%%%%%%%%%%%%%%%%%%%%%%%%%%%
%% 表題
%%%%%%%%%%%%%%%%%%%%%%%%%%%%
\begin{center}
{\LARGE SlackBotプログラム 仕様書}
\end{center}

\begin{flushright}
  2018/4/24\\
  高橋 桃花
\end{flushright}
%%%%%%%%%%%%%%%%%%%%%%%%%%%%
%% 概要
%%%%%%%%%%%%%%%%%%%%%%%%%%%%
\section{概要}
\label{sec:introduction}
本資料は,平成30年度B4新人研修課題のSlackBotプログラムの仕様についてまとめたものである.本プログラムで使用するSlack\cite{slack}とは,チャットツールである.SlackBotとは,Slackのチャットにおいて発言したり,ユーザが特定の文字列を入力すると自動で返信したりするプログラムである.本プログラムは以下の2つの機能をもつ.

\begin{enumerate}
\item 入力された文字列を返信する機能
\item 入力された飲食店の情報を返信する機能
\end{enumerate}
本資料において引用符``''に囲まれた文字列は,Slackにおける発言を示す.また,本資料における返信とは,ユーザの発言に対し,本プログラムが応答することを示す.

\section{対象とする利用者}
\label{sec:target_users}
本プログラムは以下のアカウントを所有する利用者を対象としている.

\begin{enumerate}
\item Slackアカウント
\item Googleアカウント
\end{enumerate}
Googleアカウントは本プログラムで使用するAPIキーの取得に必要である.

\section{機能}
\label{sec:function}
本プログラムは,Slackでの``@TakaBot''という文字列から始まる発言を受信し,この発言に対して返信する.返信の内容は``@TakaBot''以降の文字列により決定される.以下に本プログラムがもつ2つの機能について述べる.

\begin{description}
\item[(機能1)]  入力された文字列を返信する機能 \\
  ユーザが``@TakaBot 「(指定された文字列)」と言って''と発言した場合,本プログラムは鉤括弧内の文字列を返信する.たとえば,ユーザが``@TakaBot 「こんにちは」と言って''と発言した場合,本プログラムは\figref{fig1}のように返信する.
  \insertfigure[0.7]{fig1}{fig1}{``@TakaBot 「こんにちは」と言って''に対する返信}

\item[(機能2)]  入力された飲食店の情報を返信する機能 \\
  ユーザが``@TakaBot 「(飲食店の名前)」の情報''と発言した場合,本プログラムは鉤括弧内で入力された飲食店の詳細情報を返信する.ただし,入力された飲食店の場所を特定するため,入力する文字列には,都道府県名,市町村名,駅名,および店舗名などを含む必要がある.飲食店の詳細情報はGoogle Places API\cite{placesapi}を利用して取得している.Google Places APIにより,Googleマップ\cite{googlemaps},Googleプレイス\cite{googleplace},およびGoogle+\cite{googleplus}のデータベースにアクセスし,データを取得できる.本機能により返信される内容の詳細は以下のとおりである.
  
  \begin{enumerate}
  \item 飲食店の名前 \\
    入力された飲食店名から,最も適切な飲食店の情報をGoogle Places APIにより取得する.これにより,取得した飲食店の名前を表示する.
  \item 開店ステータス \\
    ユーザが本プログラムに対する発言を入力した時点で,(1)の飲食店が営業中であるか休業中であるかを表示する.
  \item 価格帯 \\
    (1)の飲食店の価格帯を0から4の数字で表示する.詳細は以下のとおりである.
    \begin{enumerate}
    \item 0: 無料
    \item 1: 安い
    \item 2: 普通
    \item 3: やや高い
    \item 4: 非常に高い
    \end{enumerate}
  \item 評価 \\
    Googleプレイスのレビューの集計に基づき,(1)の飲食店の評価を表示する.評価は1.0から5.0の数字で表される.
  \item URL \\
    (1)の飲食店の公式WebサイトのURLを表示する.
  \item レビュー \\
    Google Places APIにより取得した,(1)の飲食店に関するレビューを1件表示する.
  \item 写真 \\
    Google Places APIにより取得した,(1)の飲食店に関する写真を1枚表示する.
  \end{enumerate}
  たとえば,ユーザが``@TakaBot 「スターバックス 岡山」の情報''と発言した場合,本プログラムは\figref{fig2}のように返信する.
  \insertfigure[0.9]{fig2}{fig2}{``@TakaBot 「スターバックス 岡山」の情報''に対する返信}
  
\end{description}
上記の(機能1),(機能2)のどちらにも当てはまらない発言を受信した場合,本プログラムは\figref{fig3}のように返信する.

\insertfigure[0.7]{fig3}{fig3}{(機能1),(機能2)のどちらにもあてはまらない場合}

\section{動作環境}
\label{sec:devenv}
本プログラムはHeroku\cite{heroku}上で動作させる.Herokuとは,アプリケーションを実行するためのプラットフォームである.なお,本プログラムではHerokuのフリープランを利用している.以下,\tabref{tab:heroku_env}にHerokuの環境を示す.

\begin{table}[h]
  \begin{center}
    \caption{動作環境(Heroku)}\label{tab:heroku_env}
    \begin{tabular}{l|l}
      \hline \hline
      項目 & 内容 \\ \hline
      OS & Ubuntu 16.04.4 LTS\\ 
      CPU & Intel(R) Xeon(R) CPU E5-2670 v2 @ 2.50GHz\\
      メモリ & 512MB \\
      Ruby & 2.5.1p57 \\
      Ruby Gem & bundler 1.16.1 \\
      & mustermann 1.0.2\\
      & rack 2.0.4 \\
      & rack-protection 2.0.1\\
      & sinatra 2.0.1\\
      & tilt 2.0.8 \\
      \hline
    \end{tabular}
  \end{center}
\end{table}

\newpage
\section{環境構築}
\label{sec:setup}
\subsection{概要}
本プログラムの動作のために必要な環境構築の項目を以下に示す.

\begin{enumerate}
\item Herokuの設定
\item SlackのIncoming WebHooksの設定
\item SlackのOutgoing WebHooksの設定
\item Google Places APIのAPIキー取得
\end{enumerate}
次節で各項目における具体的な手順について述べる.

\subsection{手順}
\subsubsection{Herokuの設定}
\begin{enumerate}
\item 以下のURLよりHerokuにアクセスし,「Sign up」から新しいアカウントを登録する.\\
https://www.heroku.com/

\item 登録したアカウントでログインし,「Getting Started with Heroku」の使用する言語として「Ruby」を選択する.

\item 「I'm ready to start」をクリックし,「Download Heroku CLI for...」からCLI(Command Line Interface)をダウンロードする.

\item ターミナルで以下のコマンドを実行し,Heroku CLIがインストールされたことを確認する.
\begin{verbatim}
$ heroku version
heroku-cli/6.16.11-b6217f5 (linux-x64) node-v9.11.1
\end{verbatim}

\item 以下のコマンドを実行し,Herokuにログインする.
\begin{verbatim}
$ heroku login
\end{verbatim}

\item 本プログラムのディレクトリに移動して以下のコマンドを実行し,Heroku上にアプリケーションを生成する.
\begin{verbatim}
$ heroku create <app_name>
\end{verbatim}
ただし,\verb|<app_name>|は任意のアプリケーション名を示す.アプリケーション名には英語のアルファベットの小文字,数字,およびハイフンのみ使用できる.

\item 以下のコマンドを実行し,生成したアプリケーションがリモートリポジトリに登録されていることを確認する.
\begin{verbatim}
$ git remote -v
heroku https://git.heroku.com/<app_name>.git (fetch)
heroku https://git.heroku.com/<app_name>.git (push)
\end{verbatim}
\end{enumerate}

\subsubsection{SlackのIncoming WebHooksの設定}
\begin{enumerate}
\item 以下のURLにアクセスする.\\
  https://$<$team\_name$>$.slack.com/apps/manage/custom-integrations\\
  ただし,$<$team\_name$>$は自分のチーム名を示す.
\item 「Incoming WebHooks」をクリックする.
\item 「Add Configuration」をクリックする.
\item 「Choose a channel...」から発言を投稿したいチャンネルを選択し,「Add Incoming WebHooks integration」をクリックする.
\item Webhook URLを取得する.以下のコマンドにより取得したURLをHerokuの環境変数に設定する.
\begin{verbatim}
  $ heroku config:set INCOMING_WEBHOOK_URL="https://<webhook_url>"
\end{verbatim}
ただし,\verb|<webhook_url>|は取得した自分のWebhook URLを示す.
\end{enumerate}

\subsubsection{SlackのOutgoing WebHooksの設定}
\begin{enumerate}
\item 以下のURLにアクセスする.\\
 https://$<$team\_name$>$.slack.com/apps/manage/custom-integrations\\
  ただし,$<$team\_name$>$は自分のチーム名を示す.
\item 「Outgoing WebHooks」をクリックする.
\item 「Add Configuration」をクリックする.
\item 「Add Outgoing WebHooks integration」をクリックし,以下の項目を設定する.

\begin{enumerate}
  \item Channelにて,発言を監視するチャンネルを選択する.
  \item Trigger Word(s)に,WebHooksが動作する契機となる単語を設定する.
  \item URL(s)に,WebHooksが動作した際にPOSTするURLを設定する.本プログラムは
  Heroku上で動作させるため以下のURLを設定する.\\
  https://$<$app\_name$>$.herokuapp.com/slack \\
  ただし,$<$app\_name$>$はHerokuに登録したアプリケーション名を示す.
\end{enumerate}
\end{enumerate}

\subsubsection{Google Places APIのAPIキー取得}
\begin{enumerate}
\item 以下のURLにアクセスし,「キーの取得」をクリックする.\\
  https://developers.google.com/places/web-service
\item 「Create a new project」を選択し,プロジェクト名を決定する.
\item 「Next」をクリックするとAPIキーが生成される.
\item 以下のコマンドを実行し,Herokuの環境変数に取得したAPIキーを設定する.
\begin{verbatim}
$ heroku config:set GOOGLE_PLACES_APIKEY="API key"
\end{verbatim}
\end{enumerate}

\section{使用方法}
\label{usage}
本プログラムの使用方法について述べる.本プログラムはHeroku上で動作するため,Herokuへデプロイすることで実行できる.以下のコマンドを用いてHerokuにデプロイする.

\begin{verbatim}
$ git push heroku master
\end{verbatim}

\section{エラー処理と保証しない動作}
\label{error_handling}
本プログラムにおけるエラー処理と保証しない動作について述べる.

\subsection{エラー処理}
\begin{enumerate}
\item (機能2)について,本プログラムが,入力された飲食店の情報をGoogle Places APIから取得できなかった場合,\figref{fig4}のようにユーザに返信する.
\insertfigure[0.5]{fig4}{fig4}{Google Places APIから情報を取得できなかった場合}

\end{enumerate}

\subsection{保証しない動作}
本プログラムが保証しない動作を以下に示す.

\begin{enumerate}
\item SlackのOutgoing WebHooks以外からのPOSTリクエストをブロックする動作
\end{enumerate}

\bibliographystyle{ipsjunsrt}
\bibliography{mybibfile}

\end{document}
