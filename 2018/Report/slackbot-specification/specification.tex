\documentclass[12pt]{jsarticle}
\usepackage[dvipdfmx]{graphicx}
\textheight = 25truecm
\textwidth = 18truecm
\topmargin = -1.5truecm
\oddsidemargin = -1truecm
\evensidemargin = -1truecm
\marginparwidth = -1truecm

\def\theenumii{\Alph{enumii}}
\def\theenumiii{\alph{enumiii}}
\def\labelenumi{(\theenumi)}
\def\labelenumiii{(\theenumiii)}
\def\theenumiv{\roman{enumiv}}
\def\labelenumiv{(\theenumiv)}
\usepackage{comment}

%%%%%%%%%%%%%%%%%%%%%%%%%%%%%%%%%%%%%%%%%%%%%%%%%%%%%%%%%%%%%%%%
%% sty/ にある研究室独自のスタイルファイル
\usepackage{jtygm}  % フォントに関する余計な警告を消す
\usepackage{nutils} % insertfigure, figref, tabref マクロ

\def\figdir{./figs} % 図のディレクトリ
\def\figext{pdf}    % 図のファイルの拡張子

\begin{document}
%%%%%%%%%%%%%%%%%%%%%%%%%%%%
%% 表題
%%%%%%%%%%%%%%%%%%%%%%%%%%%%
\begin{center}
{\LARGE SlackBotプログラム 仕様書}
\end{center}

\begin{flushright}
  2018/4/24\\
  高橋 桃花
\end{flushright}
%%%%%%%%%%%%%%%%%%%%%%%%%%%%
%% 概要
%%%%%%%%%%%%%%%%%%%%%%%%%%%%
\section{概要}
\label{sec:introduction}
本資料は,平成30年度B4新人研修課題のSlackBotプログラムの仕様についてまとめたものである.本プログラムで使用するSlack\cite{slack}とは,チャットツールである.SlackBotとは,Slackにおけるやり取りを自動化するプログラムである.本プログラムは以下の2つの機能をもつ.

\begin{enumerate}
\item 入力された文字列を返信する機能
\item 入力された飲食店の情報を返信する機能
\end{enumerate}

また,本資料において引用符``''に囲まれた文字列は,Slackにおける発言を示す.

\section{対象とする利用者}
\label{sec:target_users}
本プログラムは以下のアカウントを所有する利用者を対象としている.

\begin{enumerate}
\item Slackアカウント
\item Googleアカウント
\end{enumerate}
Googleアカウントは本プログラムで使用するAPIキーの取得に必要である.

\section{機能}
\label{sec:function}
本プログラムは,Slackでの``@TakaBot''という文字列から始まる発言を受信し,その発言に対して返信する.返信の内容は``@TakaBot''以降の文字列により決定される.以下に本プログラムがもつ2つの機能について述べる.
\begin{description}
\item[(機能1)]  入力された文字列を返信する機能 \\
  ユーザが``@TakaBot 「(指定された文字列)」と言って''と発言した場合,本プログラムは鉤括弧内の文字列を返信する.例えば,ユーザが``@TakaBot 「こんにちは」と言って''と発言した場合,本プログラムは``こんにちは''と返信する.

\item[(機能2)]  入力された飲食店の情報を返信する機能 \\
  ユーザが``「(飲食店の名前)」の情報''と発言した場合,本プログラムは鉤括弧内で指定された飲食店の詳細情報を返信する.詳細情報は以下の通りである.
\newpage
  \begin{enumerate}
  \item 飲食店の名前
  \item 開店ステータス
  \item 価格帯
  \item 評価
  \item URL
  \item 最新のレビュー
  \item 投稿写真
  \end{enumerate}
  以上の情報はGoogle Places API\cite{placesapi}を利用して取得している.
\end{description}
上記の(機能1),(機能2)のどちらにも当てはまらない発言を受信した場合,本プログラムは以下のように返信する.

\begin{verbatim}
@ユーザ名 Hi!
使い方1: 「○○」と言って,使い方2: 「飲食店の名前」の情報
\end{verbatim}

\section{動作環境}
\label{sec:devenv}
本プログラムの動作環境を\tabref{tab:prod_env}に示す.また,\tabref{tab:prod_env}の環境において本プログラムが正常に動作することを確認した.

\begin{table}[h]
  \begin{center}
    \caption{動作環境}\label{tab:prod_env}
    \begin{tabular}{l|l}
      \hline \hline
      項目 & 内容 \\ \hline
      OS & Debian GNU/Linux 8 (jessie) 64-bit \\ 
      CPU & Intel Core i5-4670 CPU (3.40GHz)\\
      メモリ & 1.0GB \\ 
      ブラウザ & FireFox 59.0.2 \\ 
      ソフトウェア & Ruby 2.5.1 \\
      & bundler 1.16.1 \\
      & heroku CLI 6.16.11-b6217f5  \\
      & Git 2.1.4 \\
      \hline
    \end{tabular}
  \end{center}
\end{table}

また,本プログラムに必要なGemを\tabref{tab:gem_table}に示す.Gemとは,Rubyで使用することのできるライブラリである.

\begin{table}[h]
  \begin{center}
    \caption{本プログラムに必要なGem}\label{tab:gem_table}
    \begin{tabular}{l|l}
      \hline \hline
      Gem & バージョン \\
      \hline 
      mustermann & 1.0.2 \\
      rack & 2.0.4 \\
      rack-protection & 2.0.1 \\
      sinatra & 2.0.1 \\
      tilt & 2.0.8 \\
      \hline
    \end{tabular}
  \end{center}
\end{table}


\newpage
\section{環境構築}
\label{sec:setup}
\subsection{概要}
本プログラムの動作のために必要な環境構築の項目を以下に示す.

\begin{enumerate}
\item Herokuの設定
\item SlackのIncoming WebHooksの設定
\item SlackのOutgoing WebHooksの設定
\item Google Places APIのAPIキー取得
\item Gemのインストール
\end{enumerate}
次節で各項目における具体的な手順について述べる.

\subsection{手順}
\subsubsection{Herokuの設定}
\begin{enumerate}
\item 以下のURLよりHerokuにアクセスし,「Sign up」から新しいアカウントを登録する.\\
https://www.heroku.com/

\item 登録したアカウントでログインし,「Getting Started with Heroku」の使用する言語として「Ruby」を選択する.

\item 「I'm ready to start」をクリックし,「Download Heroku CLI for...」からCLI(Command Line Interface)をダウンロードする.

\item ターミナルで以下のコマンドを実行し,Heroku CLIがインストールされたことを確認する.
\begin{verbatim}
$ heroku version
heroku-cli/6.16.11-b6217f5 (linux-x64) node-v9.11.1
\end{verbatim}

\item 以下のコマンドを実行し,Herokuにログインする.
\begin{verbatim}
$ heroku login
\end{verbatim}

\item 本プログラムのディレクトリに移動して以下のコマンドを実行し,Heroku上にアプリケーションを生成する.
\begin{verbatim}
$ heroku create <myapp_name>
\end{verbatim}
ただし,\verb|<myapp_name>|は任意のアプリケーション名を示す.アプリケーション名には小文字,数字,およびハイフンのみ使用できる.

\item 以下のコマンドを実行し,生成したアプリケーションがリモートリポジトリに登録されていることを確認する.
\begin{verbatim}
$ git remote -v
heroku https://git.heroku.com/<myapp_name>.git (fetch)
heroku https://git.heroku.com/<myapp_name>.git (push)
\end{verbatim}
\end{enumerate}

\subsubsection{SlackのIncoming WebHooksの設定}
\begin{enumerate}
\item 以下のURLにアクセスする.\\
  https://XXXXX.slack.com/apps/manage/custom-integrations\\
  ただし,XXXXXは自分のチーム名を示す.
\item 「Incoming WebHooks」をクリックする.
\item 「Add Configuration」をクリックする.
\item 「Choose a channel...」から発言を投稿したいチャンネルを選択し,「Add Incoming WebHooks integration」をクリックする.
\item WebHook URLを取得する.以下のコマンドにより取得したURLをHerokuの環境変数に設定する.
\begin{verbatim}
  $ heroku config:set INCOMING_WEBHOOK_URL="https://XXXXX"
\end{verbatim}
ただし,\verb|XXXXX|は取得した自分のWebHook URLを示す.
\end{enumerate}

\subsubsection{SlackのOutgoing WebHooksの設定}
\begin{enumerate}
\item 以下のURLにアクセスする.\\
 https://XXXXX.slack.com/apps/manage/custom-integrations\\
  ただし,XXXXXは自分のチーム名を示す.
\item 「Outgoing WebHooks」をクリックする.
\item 「Add Configuration」をクリックする.
\item 「Add Outgoing WebHooks integration」をクリックし,以下の項目を設定する.
\newpage
\begin{enumerate}
  \item Channelにて,発言を監視するチャンネルを選択する.
  \item Trigger Word(s)に,WebHooksが動作する契機となる単語を設定する.
  \item URL(s)に,WebHooksが動作した際にPOSTするURLを設定する.本プログラムは
  Heroku上で動作させるため以下のURLを設定する.\\
  https://XXXXX.herokuapp.com/slack \\
  ただし,XXXXXはHerokuに登録したアプリケーション名を示す.
\end{enumerate}
\end{enumerate}

\subsubsection{Google Places APIのAPIキー取得}
\begin{enumerate}
\item 以下のURLにアクセスし,「キーの取得」をクリックする.\\
  https://developers.google.com/places/web-service
\item 「Create a new project」を選択し,プロジェクト名を決定する.
\item 「Next」をクリックするとAPIキーが生成される.
\item 以下のコマンドを実行し,Herokuの環境変数に取得したAPIキーを設定する.
\begin{verbatim}
$ heroku config:set GOOGLE_PLACES_APIKEY="API key"
\end{verbatim}
\end{enumerate}

\subsubsection{Gemのインストール}
本プログラムで使用するGemをbundlerを用いてインストールする.bundlerとは,Gemの依存関係やバージョンを管理するためのものである.
\begin{enumerate}
\item 以下のコマンドを実行し,Gemをインストールする.
\begin{verbatim}
$ bundle install --path vendor/bundle
\end{verbatim}
\end{enumerate}

\section{使用方法}
\label{usage}
本プログラムの仕様方法について述べる.本プログラムはHeroku上で動作するため,Herokuへデプロイすることで実行できる.以下のコマンドを用いてHerokuにデプロイする.

\begin{verbatim}
$ git push heroku master
\end{verbatim}

\newpage
\section{エラー処理と保証しない動作}
\label{error_handling}
本プログラムにおけるエラー処理と保証しない動作について述べる.

\subsection{エラー処理}
\begin{enumerate}
\item (機能2)について,本プログラムが指定された飲食店の情報をGoogle Places APIから取得できなかった場合,以下のようにユーザに返信する.
\begin{verbatim}
結果が取得できませんでした
\end{verbatim}
\end{enumerate}

\subsection{保証しない動作}
本プログラムが保証しない動作を以下に示す.

\begin{enumerate}
\item SlackのOutgoing WebHooks以外からのPOSTリクエストをブロックする動作
\end{enumerate}

\bibliographystyle{ipsjunsrt}
\bibliography{mybibfile}

\end{document}
